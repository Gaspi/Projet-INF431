\documentclass[12pt,a4paper,titlepage]{article}
\usepackage[utf8]{inputenc}


\usepackage[left=2cm,right=2cm,top=2cm,bottom=2cm]{geometry}
\usepackage{amsmath}
\usepackage{amssymb}
\usepackage{amsthm}
\usepackage{hyperref}

\author{Jonathan Visbecq, Gaspard Férey}
\title{Projet d'INF 431 \\ - \\ Notice d'utilisation}

\usepackage{listings}
\lstset{language=Java}

\newcommand{\class}[1]{$\mathtt{#1}$}



\begin{document}
\maketitle

\section{Introduction}
Vous venez de télécharger notre solution au projet du cours d'INF 431 de seconde année à l'école Polytechnique de Paris. Ce projet est proposé par monsieur Bruno Salvy et disponible à \href{http://perso.ens-lyon.fr/bruno.salvy/INF431/Projet/INF431_-_Projet_Informatique.html}{cette adresse}.
L'intégralité de la solution que nous proposons est constitué de
\begin{itemize}
\item "Notice d'utilisation.pdf" : La présente notice
\item "Rapport.pdf" : Le rapport du projet. Vous pourrez y trouver une description plus détaillée des algorithmes employés. 
\item "executable.jar" : La version exécutable aboutie du projet, utilisable immédiatement.
\item "bin" : Le dossier dans lequel sont situés les fichier ".class" des classes utilisées.
\item "workspace" : Le dossier à utiliser comme "workspace" pour ouvrir le projet à l'aide d'Eclipse. Il contient le code source de notre projet.
\item "files.rar" : Un dossier compressé dans lequel sont situés quelques fichiers volumineux pouvant être utilisés pour tester les programmes.
\end{itemize}


\section{Pour une utilisation rapide}
- Placez dans un même dossier l'exécutable "executable.jar" et un dossier intitulé "files".\\
- Placez dans ce sous-dossier "files" les fichiers que vous souhaitez traiter. \\
- Lancez "executable.jar" et suivez les instructions à l'écran.



\section{Utiliser le code source}
Nos programmes peuvent également être utilisés directement à l'aide du code compilé.
\begin{itemize}
\item Placez dans un même dossier le dossier "bin" et un dossier intitulé "files".
\item Placez dans ce sous-dossier "files" les fichiers que vous souhaitez traiter.
\item En ligne de commande, placez-vous dans le dossier "bin".
\item Exécutez les commandes suivantes selon la fonction qui vous intéresse (cf ci-dessous).
\end{itemize}

Remarques :
\begin{itemize}
\item[-] les adresses des fichiers peuvent être
	\begin{itemize}
	\item soit leur adresse absolue
	\item soit l'adresse relative à partir du répertoire actuel, "bin".
	\end{itemize}
\item[-] Les mots clef désignant une fonctions de hachage doivent être choisis parmi :
	\begin{center}
	LookUp3 \\
	MurmurHash3 \\
	JavaHash \\
	LoseLose \\
	HomemadeHash \\
	DJB2
	\end{center}
\end{itemize}



\subsection{Filtrage}
\begin{lstlisting}
java filter/Filter origine cible
\end{lstlisting}
\class{origine} désigne l'adresse d'un fichier existant à filtrer.\\
\class{cible} désigne l'adresse du fichier filtré à créer.\\


\subsection{Hachage}

\subsection{Comptage approché}

\subsection{Similarités entre ensembles de données}

\subsection{Fenêtre glissante}

\subsection{Échantillonnage}

\subsection{Souris}

\subsection{Icebergs}

\subsection{Récapitulatif}
\begin{center} \begin{tabular}{ll}
0. Filtrage 		 & filter/Filter \\
1. Hachage 			 & hash/HashFunctionTests \\
2. Comptage approché & drafts/HyperLogLog \\
3. Similarités entre ensembles de données & drafts/Similarities \\
4. Fenêtre glissante & drafts/SlidingWindow \\
5. Échantillonnage 	 & sampling/Sampling \\
6. Souris 			 & sampling/Mice \\
7. Icebergs 		 & sampling/Icebergs
\end{tabular} \end{center}





\section{Utiliser Eclipse}
- Ouvrez Eclipse\\
- Sélectionnez : File -$>$ Switch Worspace -$>$ Other...\\
- Sélectionnez l'emplacement du dossier "workspace"\\
- Réglez le niveau de compatibilité à "Java 1.7"\\
- Parcourez l'intégralité du code source et appelez les différentes fonctions à partir de la classe "Main.java"


\end{document}