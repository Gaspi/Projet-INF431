\documentclass[12pt,a4paper,titlepage]{article}
\usepackage[utf8]{inputenc}


\usepackage[left=2cm,right=2cm,top=2cm,bottom=2cm]{geometry}
\usepackage{amsmath}
\usepackage{amssymb}
\usepackage{amsthm}
\usepackage{hyperref}

\author{Jonathan Visbecq, Gaspard Férey}
\title{Projet d'INF 431 \\ - \\ Notice d'utilisation}


\begin{document}
\maketitle

\section{Introduction}
Vous venez de télécharger notre solution au projet du cours d'INF 431 de seconde année à l'école Polytechnique de Paris. Ce projet est proposé par monsieur Bruno Salvy et disponible à \href{http://perso.ens-lyon.fr/bruno.salvy/INF431/Projet/INF431_-_Projet_Informatique.html}{cette adresse}.
Notre groupe est composé de Gaspard Férey et Jonathan Visbecq.\\
L'intégralité de la solution que nous proposons est constitué de
\begin{itemize}
\item "Notice d'utilisation.pdf" : La présente notice
\item "Rapport.pdf" : Le rapport du projet. Vous pourrez y trouver une description plus détaillée des algorithmes employés. 
\item "executable.jar" : La version exécutable aboutie du projet, utilisable immédiatement.
\item "code source" : Le dossier dans lequel sont situé les fichier ".java" des classes utilisées.
\item "workspace" : Le dossier à utiliser comme "workspace" pour ouvrir le projet à l'aide d'Eclipse.
\item "files.rar" : Un dossier compressé dans lequel sont situés des fichiers volumineux pouvant être utilisés pour tester les programmes.
\end{itemize}


\section{Pour une utilisation rapide}
- Placez dans un même dossier l'exécutable "executable.jar" et un dossier intitulé "files".\\
- Placez dans ce sous-dossier "files" les fichiers que vous souhaitez traiter. \\
- Lancez "executable.jar" et suivez les instructions à l'écran.


\section{Utiliser le code source}
Nos programmes peuvent également être compilés et utilisés directement à l'aide du code source.
- Compilez l'ensemble des fichiers du projet.\\
- Exécutez les fichiers correspondants à la question qui vous intéresse.
\begin{center} \begin{tabular}{ll}
0. Filtrage 		 & filter/Filter \\
1. Hachage 			 & hash/HashFunctionTests \\
2. Comptage approché & drafts/HyperLogLog \\
3. Similarités entre ensembles de données & drafts/Similarities \\
4. Fenêtre glissante & drafts/SlidingWindow \\
5. Échantillonnage 	 & sampling/Sampling \\
6. Souris 			 & sampling/Mice \\
7. Icerbergs 		 & sampling/Icebergs
\end{tabular} \end{center}

\section{Utiliser Eclipse}
- Ouvrez Eclipse\\
- Sélectionnez : File -$>$ Switch Worspace -$>$ Other...\\
- Sélectionnez l'emplacement du dossier "workspace"\\
- Parcourez l'intégralité du code source et appelez les différentes fonctions à partir de la classe "Main.java"



\end{document}