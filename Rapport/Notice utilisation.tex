\documentclass[12pt,a4paper,titlepage]{article}
\usepackage[utf8]{inputenc}


\usepackage[left=2cm,right=1.5cm,top=1.5cm,bottom=1.5cm]{geometry}
\usepackage{amsmath}
\usepackage{amssymb}
\usepackage{amsthm}
\usepackage{hyperref}

\author{Jonathan Visbecq, Gaspard Férey}
\title{Projet d'INF 431 \\ - \\ Notice d'utilisation}

\newcommand{\code}[1]{ \texttt{\footnotesize #1} }

\begin{document}
\maketitle

\section{Introduction}
Vous venez de télécharger notre solution au projet du cours d'informatique INF 431 de seconde année à l'école Polytechnique de Paris. Ce projet est proposé par monsieur Bruno Salvy et disponible à \href{http://perso.ens-lyon.fr/bruno.salvy/INF431/Projet/INF431_-_Projet_Informatique.html}{cette adresse}.\\
L'intégralité de la solution que nous proposons est constitué des fichiers et dossiers suivants :
\begin{itemize}
\item "Notice d'utilisation.pdf" : La présente notice.
\item "Sujet.pdf" : Le sujet proposé par monsieur Bruno Salvy.
\item "Rapport.pdf" : Le rapport du projet. Vous pourrez y trouver une description plus détaillée des algorithmes employés. 
\item "executable.jar" : L'exécutable java qui lance l'interface graphique (GUI) du projet. Permet d'utiliser rapidement les fonctionnalités développées dans ce projet.
\item "bin" : Le dossier dans lequel sont situés les fichiers java compilés ".class" des classes utilisées.
\item "workspace" : Le dossier à utiliser comme "Java Workspace" pour ouvrir le projet à l'aide d'Eclipse. Il contient le code source de notre projet (sans la GUI).
\item "files.rar" : Un dossier compressé dans lequel sont situés quelques fichiers volumineux pouvant être utilisés pour tester les programmes.
\end{itemize}

\newpage
\section{Pour une utilisation rapide}
- Placez dans un même dossier l'exécutable "executable.jar" et un dossier intitulé "files".\\
- Placez dans ce sous-dossier "files" les fichiers que vous souhaitez traiter. \\
- Lancez "executable.jar" et suivez les instructions à l'écran.



\section{Utiliser le code source}
Nos programmes peuvent également être utilisés directement à l'aide du code compilé.
\begin{itemize}
\item Placez dans un même dossier le dossier "bin" et un dossier intitulé "files".
\item Placez dans ce sous-dossier "files" les fichiers que vous souhaitez traiter.
\item En ligne de commande, placez-vous dans le dossier "bin".
\item Exécutez les commandes suivantes selon la fonction qui vous intéresse (cf ci-dessous).
\end{itemize}

\textbf{Remarques :}
\begin{itemize}
\item[-] les adresses des fichiers peuvent être
	\begin{itemize}
	\item soit leur adresse absolue
	\item soit l'adresse relative à partir du répertoire actuel : "bin".
	\end{itemize}
\item[-] Les mots clef désignant une fonctions de hachage doivent être choisis parmi :
	\begin{center}
	LookUp3 \\
	MurmurHash3 \\
	JavaHash \\
	LoseLose \\
	HomemadeHash \\
	DJB2
	\end{center}
\item[-] Tout fonction peut être appelée \emph{sans argument}. Une interface en ligne de commande vous proposera alors de rentrer les arguments un par un.
\end{itemize}



\subsection{Filtrage}
Cette fonction permet d'appliquer un filtre sur un fichier écrit en anglais. Les différentes fonctions du projet sont conçues pour fonctionner sur des fichiers filtrés.\\
\indent\indent\code{java filter/Filter origine cible}\\
\begin{tabular}{lcl}
\code{origine} &:& l'adresse d'un fichier existant à filtrer.\\
\code{cible} &:& l'adresse à laquelle enregistrer le fichier filtré créé.
\end{tabular}\\


\subsection{Hachage}
Cette fonction affiche les résultats des tests de vitesse et de collisions appliqués aux fonctions de hachage souhaitée.\\
\indent\indent\code{java hash/HashFunctionTests file1 file2 ...}\\
\code{file1}, \code{file2}, ... \ : \ les adresses des fichiers filtrés sur lesquels effectuer les tests.\\


\subsection{Comptage approché}
Cette fonction applique l'algorithme \emph{HyperLogLog} à un fichier donné en utilisant une fonction de hachage au choix.\\
\indent\indent\code{java hyperLogLog/HyperLogLog hash b file}\\
\begin{tabular}{lcl}
\code{file} &:& l'adresse du fichier filtré cible.\\
\code{b} &:& le paramètre b de l'algorithme (4 - 15).\\
\code{hash} &:& la fonction de hachage à employer.
\end{tabular}\\


\subsection{Similarités entre ensembles de données}
Cette fonction calcule les similarités entre différents fichiers et affiche les fichiers suffisamment semblables. \\
\indent\indent\code{java hyperLogLog/Similarities hash b k seuil file1 file2 ...}\\
\begin{tabular}{lcl}
\code{hash} &:& la fonction de hachage à employer.\\
\code{b} &:& le paramètre $b$ de l'algorithme (4 - 15).\\
\code{k} &:& le paramètre $k$ de l'algorithme (1 - 50).\\
\code{file1}, \code{file2}, ... &:& les adresses des fichiers filtré à comparer.\\
\code{seuil} &:& seules les paires de fichiers dont la "similarité" est supérieure\\
&& à ce seuil sont affichées (0 - 1).
\end{tabular}\\


\subsection{Fenêtre glissante}
Cette fonction affiche l'évolution, au cours de la lecture d'un fichier, du nombre de mots différents dans une fenêtre glissante de taille donnée. \\
\code{java -cp bin:./jmatharray.jar:./jmathplot.jar hyperLogLog.SlidingWindow hash b size pas file}\\
\begin{tabular}{lcl}
\code{hash} &:& la fonction de hachage à employer.\\
\code{b} &:& le paramètre $b$ de l'algorithme (4 - 15).\\
\code{size} &:& la taille de la fenêtre (1 - 1000000).\\
\code{pas}&:& le pas d'affichage du résultat. Une valeur de 1 peut nuire à la rapidité\\
&& de l'affichage selon la taille du fichier choisit (1 - 1000000)\\
\code{file} &:& l'adresse du fichier filtré à utiliser.
\end{tabular}\\


\subsection{Échantillonnage}
Cette fonction isole quelques mots significatifs d'un texte. Ces mots peuvent, par exemple, être utilisés pour retrouver des références vers ce texte via une recherche Google. \\
\code{java  sampling/SignificantWords hash nbWords k file}\\
\begin{tabular}{lcl}
\code{hash} &:& la fonction de hachage à employer.\\
\code{nbWords} &:& le nombre de mots à choisir parmi ceux du texte (1 - 1000).\\
\code{k} &:& le paramètre $k$ de l'algorithme (1 - 1000000).\\
\code{file} &:& l'adresse du fichier filtré à utiliser.
\end{tabular}\\


\newpage
\subsection{Souris}
Cette fonction affiche l'ensemble des "souris" d'un texte.\\
\code{java  sampling/Mice hash number k file}\\
\begin{tabular}{lcl}
\code{hash} &:& la fonction de hachage à employer.\\
\code{number} &:& le nombre de "souris" que l'on souhaite obtenir (1 - 1000).\\
\code{k} &:& le paramètre $k$ de l'algorithme, (1 - 1000000000).\\
\code{file} &:& l'adresse du fichier filtré à utiliser.
\end{tabular}\\


\subsection{Icebergs}
Cette fonction affiche l'ensemble des "icebergs" d'un texte.\\
\code{java  sampling/Icebergs frequency file}\\
\begin{tabular}{lcl}
\code{frequency} &:& la fréquence limite d'apparition pour qu'un mot soit considéré comme un iceberg.\\
\code{file} &:& l'adresse du fichier filtré à utiliser.
\end{tabular}\\


\subsection{Récapitulatif}
\begin{center} \begin{tabular}{ll}
0. Filtrage 		 & \code{filter/Filter} \\
1. Hachage 			 & \code{hash/HashFunctionTests} \\
2. Comptage approché & \code{drafts/HyperLogLog} \\
3. Similarités entre ensembles de données & \code{drafts/Similarities} \\
4. Fenêtre glissante & \code{drafts/SlidingWindow} \\
5. Échantillonnage 	 & \code{sampling/Sampling} \\
6. Souris 			 & \code{sampling/Mice} \\
7. Icebergs 		 & \code{sampling/Icebergs}
\end{tabular} \end{center}




\section{Utiliser Eclipse}
- Ouvrez Eclipse\\
- Sélectionnez : File -$>$ Switch Worspace -$>$ Other...\\
- Sélectionnez l'emplacement du dossier "workspace"\\
- Parcourez l'intégralité du code source et appelez les différentes fonctions à partir de la classe "Main.java"

\end{document}